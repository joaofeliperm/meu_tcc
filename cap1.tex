\chapter{INTRODUÇÃO}

A Tecnologia da Informação (TI) pode ser definida como o conjunto de todas as atividades e soluções fornecidas por recursos de computação que visam permitir a produção, armazenamento, transmissão, acesso e o uso das informações \cite{WIKIPEDIA}.

É sabido que para empresas, o tratamento das suas informações são uma fonte de estratégias para lucrar ou conter desperdícios. Segundo \citeonline{LAURINDO}, a TI evoluiu de uma orientação tradicional de suporte administrativo para um papel estratégico dentro da organização . Dessa forma, fica evidente que na sociedade da informação, as modernas TI têm influenciado decisivamente as organizações, tanto as grandes quanto as pequenas empresas \cite{CRISTINA}.

Nas micro e pequenas empresas (MPE), a TI assume ainda um papel mais importante, que está relacionado ao tempo de vida no mercado. Nesse contexto, os sistemas de informação, através das várias tecnologias disponíveis, podem contribuir para a sobrevivência e desenvolvimento das MPE, constituindo-se em ferramenta estratégica para enfrentar e superar desafios \cite{VALDIR}.

A infraestrutura de TI entretanto é cara, sendo necessário viabilizar recursos com custos mais compatíveis com pequenas organizações.

A partir de 2006 tornou-se disponível o hardware raspberry pi,  que pode ser útil e apresentar um custo benefício interessante para empresas de pequeno porte.

O objetivo deste trabalho é estudar e apresentar as características deste dispositivo, avaliar seu potencial de uso, bem como exemplificar seu uso para ambientes comerciais. Desta forma será possível iniciar a avaliação do hardware em ambiente acadêmico e analisar critérios como custo, desempenho, robustez, escalabilidade e operabilidade.