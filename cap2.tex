\chapter{METODOLOGIA}



Inicialmente o professor orientador deste trabalho realizou a compra do equipamento, e em seguida buscou-se realizar testes que demonstrassem que o \textit{Raspberry Pi} seria um bom tema a ser discutido nesta monografia. Os testes foram: ligar o equipamento, analisar as funcionalidades apresentadas pelo sistema operacional e realizar tarefas simples, como acessar a internet, usar a calculadora e abrir um arquivo de extensão .pdf. Quando executados, todos os testes citados anteriormente retornaram seus resultados com tempo de resposta normal, sem apresentar lentidão.

Desta forma, iniciou-se uma revisão bibliográfica para a pesquisa e obtenção de informações sobre o \textit{Raspberry Pi} que viabilizassem a escrita de um referencial teórico, sessão responsável por apresentar uma sequência de assuntos e fornecer embasamento conceitual para o entendimento deste trabalho.

Pautando-se nos resultados dos testes comentados no primeiro parágrafo e na análise do referencial teórico, foi planejado um estudo de caso que avaliasse a utilização do \textit{Raspberry Pi} num ambiente de produção. Tal avaliação foi realizada na Assessoria de Suporte da Gerência de Informação da Universidade Estadual do Norte Fluminense Darcy Ribeiro, onde funcionários trabalham diariamente utilizando um sistema \textit{web} para gerenciar as demandas na área de tecnologia da informação.

Após o uso do equipamento pelos integrantes do setor, foi fornecido um questionário, que visa quantificar a satisfação qualitativamente, e com isso avaliar a adequação para o caso específico. O resultados do questionário foram analisados e discutidos para produzir as conclusões do trabalho.