\chapter{METODOLOGIA}

\section{Compra do equipamento}

A compra do equipamento foi realizada via internet pelo professor orientador deste trabalho. O professor optou por importar o produto, que demorou cerca de 15 dias para chegar em mãos.

\section{Testes para viabilizar estudo}

Com o produto em mãos, buscou-se realizar testes que demonstrassem que o Raspberry Pi era um bom tema a ser discutido neste trabalho monográfico. Os testes foram: ligar o equipamento, analisar as funcionalidades apresentadas pelo sistema operacional e realizar tarefas simples, como acessar a internet, usar a calculadora, ler um arquivo de extensão .pdf e etc.

Os testes apresentaram resultados satisfatórios, o que permitiu iniciar uma pesquisa sobre o equipamento na literatura.

\section{Revisão bibliográfica}

A revisão bibliográfica permitiu a pesquisa e a obtenção, na literatura, de informações sobre o Raspberry Pi que viabilizassem a escrita de um referencial teórico.

\section{Referencial teórico}

Com base nas informações recolhidas durante a revisão bibliográfica, são apresentados uma sequência de assuntos com a finalidade de fornecer um embasamento conceitual para o entendimento deste trabalho.

\section{Planejamento do estudo de caso}

Pautando-se nos resultados dos testes citados na seção 2.2 e na análise do referencial teórico, foi planejado um estudo de caso que avaliasse a utilização do Raspberry Pi num ambiente de produção. Tal avaliação será realizada na Assessoria de Suporte da Gerência de Informação da Universidade Estadual do Norte Fluminense Darcy Ribeiro, onde X pessoas trabalham semanalmente utilizando um sistema web para gerenciar as demandas na área de tecnologia da informação.

\subsection{Planejamento do questionário}

Após o uso do equipamento pelos integrantes do setor, será fornecido um questionário para a avaliação do mesmo. A análise do questionário visa quantificar a satisfação qualitativamente, e com isso avaliar a adequação para o caso específico.

\section{Discussão dos resultados}

A discussão dos dados obtidos, bem como das impressões capturadas durante o trabalho são então discutidas no tópico resultados.