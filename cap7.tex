\chapter{RESULTADOS OBTIDOS}

O \textit{Raspberry Pi} não é rápido o suficiente para realizar trabalhos que façam uso intensivo de leitura e escrita, como um servidor por exemplo. Entretanto é suficientemente rápido para aplicações standalone, quando não há um uso excessivo dos recursos do equipamento.

Seu custo é razoável para aquisição de uma primeira máquina, mas sua compra em larga escala não é viável por dificuldade de fornecimento e pela dificuldade de suporte local.

Sua utilização como estação de trabalho para ambiente web é suficientemente estável, mas ao acessar sites que demandam um uso mais intenso de memória, a navegação se torna desconfortável e demorada.

Durante o estudo de caso as funcionalidades foram testadas com sucesso, sem a ocorrência de qualquer falha, gerando nas pessoas que utilizaram o sistema uma impressão aparentemente positiva em relação ao equipamento apresentado. As respostas são apresentadas na tabela 7.1.

\begin{table}[!htpb]
 \centering
    \begin{tabular}{|c|c|c|c|c|c|c|c|c|c|} 
    \hline
        \textbf{Usuário} & \textbf{P1} & \textbf{P2} & \textbf{P3} & \textbf{P4} & \textbf{P5} & \textbf{P6} & \textbf{P7} & \textbf{P8} & \textbf{P9} \\
    \hline
        Técnico 1 & 5 & 4 & 5 & 5 & 5 & 3 & 3 & 4 & 4 \\
    \hline
        Técnico 2 & 5 & 5 & 4 & 4 & 4 & 5 & 5 & 5 & 5 \\
    \hline
        Técnico 3  & 5 & 4 &	4 &	5 &	5 &	5 &	5 &	4 &	4 \\
    \hline
    	Técnico 4  & 5 & 5 &	5 &	5 &	5 &	5 &	5 &	5 &	5 \\
    \hline
 	    Técnico 5  & 5 & 5 & 5 & 5 & 5 & 4 & 5 & 4 & 5 \\
    \hline
    \end{tabular}
    \caption{Respostas dos técnicos obtidas através do questionário}
    \label{t_fixa}
\end{table}

A primeira pergunta tinha como objetivo avaliar o nível de facilidade para colocar o \textit{Raspberry Pi} em funcionamento. A segunda e a terceira eram relacionadas ao desempenho do \textit{browser} para acessar o site da aplicação e para manipulá-lo. A quarta e a quinta capturaram do usuário se o equipamento travou quando executada algumas funcionalidades do sistema ou em qualquer outro momento.

A sexta tinha como objetivo quantificar a ideia de trocar o desktop do usuário pelo \textit{Raspberry Pi}. A sétima quantificava o uso do Pi em qualquer outro negócio que tivesse um sistema semelhante ao do setor. A oitava perguntava ao usuário se o preço do equipamento reforçava a construção de um sistema web para sua empresa, caso fosse dono de uma. E a nona concluía o questionário perguntando ao usuário se ele aprovava o equipamento de uma forma geral.

O \textit{Raspberry Pi} obteve uma aprovação total de aproximadamente 93\% dos usuários. Tal resultado corrobora com a opinião de que a experiência pode ser positiva para pequenas empresas.