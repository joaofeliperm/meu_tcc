\chapter{METODOLOGIA}

Neste capítulo é feita uma revisão de assuntos relacionados, visando embasamento conceitual, para o entendimento deste trabalho.

\section{Raspberry Pi}

\subsection{Definição}

\textit{Raspberry Pi} é um computador do tamanho de um cartão de  rédito desenvolvido no Reino Unido pela Fundação \textit{Raspberry Pi} com a intenção de estimular o ensino de informática básica nas escolas. Ele se conecta à sua TV e a um teclado, e pode ser usado para muitas das coisas que o seu PC desktop faz, como processar textos, planihas e jogos, além de reproduzir vídeo de alta definição. Todo o hardware é integrado em uma única placa, e o projeto é baseado em um \textit{system on a chip} (SoC) Broadcom BCM2835, que inclui um processador ARM1176JZF-S de 700 MHz, GPUVideoCore IV, e 512 MB de memória RAM em sua última revisão. O projeto não inclui uma memória não-volátil - como um disco rígido - mas possui uma entrada de cartão SD para armazenamento de dados. A placa é adaptada para rodar sistemas operacionais baseados em Linux \cite{WIKIPEDIA2}.

\subsection{História}

A ideia de criar um computador pequeno e barato para crianças surgiu em 2006, quando Eben Buton e seus colegas da Universidade de Cambridge perceberam que os estudantes de hoje que querem estudar ciência da computação não têm as habilidades que eles tinham na década de 1990.

Eles atribuem isso, entre outros fatores, à ascensão do computador pessoal (PC) e dos consoles de jogos que substituíram os microcomputadores que as pessoas da geração anterior aprendiam a programar. Desde que o computador tornou-se importante para todos os membros da família, o ato de deixar os membros mais jovens brincarem ou fazerem experimentos foi sendo cada vez mais desestimulado.
Mas recentemente os processadores usados em telefones celulares e tablets tornaram-se mais baratos e mais poderosos, abrindo caminho para o lançamento do Raspeberry Pi no mundo das placas de computadores ultrabaratas e úteis.

\subsection{Viabilidade no mercado brasileiro}

Para se obter um Raspberry Pi no Brasil é preciso importá-lo. No site da farnellwark, empresa que importa e revende o equipamento no país do futebol, o valor da placa Raspberry Pi - Modelo B é R\$ 176,02. O preço apresentado anteriormente é um somatório do custo real da placa com despesas referentes à importação, sendo que o frete para a entrega do equipamento em sua residência ainda não está incluso. Para se ter uma noção do aumento no valor da compra gerado pela despesas com importação, foi pesquisado o preço em um site que vendesse a mesma placa sem tais custos. No site da adafruit, o preço encontrado foi de \$ 39,95, o equivalente a R\$ 88,61, de acordo com a cotação do dólar às 12h do dia 21 de setembro de 2013, evidenciando assim uma diferença de R\$ 87,49.

\begin{table}[!htpb]
 \centering
    \begin{tabular}{|l|p{3cm}|c|} 
    \hline
        \textbf{Valor} & \textbf{Impostos} & \textbf{Valor Total} \\
    \hline
        R\$ 90,93 & R\$ 87,49 & R\$ 176,02 \\
    \hline
    \end{tabular}
    \caption{Apresentação do valor do Raspberry Pi agregado com gastos de importação para o Brasil. O valor total não inclui o frete.}
    \label{t_fixa}
\end{table}

O frete e o tempo de entrega do equipamento em residência varia de acordo com a localidade da mesma. Para uma entrega em Campos dos Goytacazes, o valor do frete cobrado foi de R\$ 32,20 e o tempo de 3 dias úteis.

Em relação a empresas que oferecem suporte à placa no Brasil não foi encontrada nenhuma informação.

\subsection{Comparativo de custos de \textit{hardware} semelhante}

Para apresentar a diferença de custo entre o Raspberry Pi e um hardware semelhante, mais precisamente um computador desktop, foram escolhidas duas empresas para serem alvos da pesquisa de preço. A primeira será a Farnell Newark, empresa já apresentada na seção anterior e que fornecerá o valor do Raspberry Pi, e a segunda será a Americamas.com, empresa que informará o preço do PC desktop mais barato existente na loja. Em ambas as empresas tal pesquisa de preço foi realizada às 14:18 horas do dia 28 de setembro de 2013.

O valor apresentado pela Farnell Newark para o Raspberry Pi foi de R\$ 176,02 mais despesas com frete, enquanto que a loja Americanas.com vende o pc desktop mais barato pelo preço de R\$ 719,00.

Ao se analisar os valores apresentados no parágrafo anterior, é preciso entender que a comparação entre os equipamentos não acontece em torno da capacidade de seus componentes, tais como poder de processamento, quantidade de memória RAM, capacidade de disco rígido, e etc, pois se assim fosse o pc desktop desbancaria a nova tecnologia que está sendo apresentada. O objetivo do comparativo está em mostrar a capacidade que o Raspberry Pi tem de apresentar muitos dos recursos fornecidos por um pc desktop por um custo bem menor.

\subsection{Sistemas computacionais que utilizam o \textit{Raspberry Pi}}

Apesar de os idealizadores do Raspberry Pi o terem criado visando uma prática maior da programação entre jovens e crianças, o equipamento tem sido utilizado por pessoas do mundo inteiro a fim de alcançarem outros objetivos. Sendo assim, foram pesquisados projetos que apresentassem exemplos do uso do equipamento em sistemas computacionais, e o resultado é apresentado no parágrafos seguintes.

Em um artigo escrito por \citeonline{PEDRO}, o Raspberry Pi é utilizado como um servidor de Virtual Private Network (VPN) ou Rede Privada Virtual. Após demonstrar passo a passo a instalação de alguns pacotes e realizar algumas configurações necessárias, o autor do artigo explica que ao se conectar a uma rede pública e necessitar da garantia que a sua conexão é segura, pode-se fazer uso do Raspberry Pi para que as conexões sejam cifradas, passando pelo seu mini PC.

Segundo \citeonline{ZACH}, é possível fazer do Raspberry Pi um servidor web barato para realizar testes ou armazenar arquivos.
Nota-se então que já se encontra exemplos de sistemas computacionais baseados no Raspberry Pi, o que multiplica as possíveis utilidades dessa nova tecnologia.

\section{Criação de um estudo de caso}

Pautando-se em todas as seções discutidas neste capítulo, e principalmente após encontrar exemplos de projetos de sistemas computacionais utilizando o Raspberry Pi alcançarem sucesso, como apresentado na seção anterior, tornou-se viável a criação de um estudo de caso para avaliar a utilização do equipamento num ambiente de produção. Tal avaliação será realizada na Assessoria de Suporte da Gerência de Informação da Universidade Estadual do Norte Fluminense Darcy Ribeiro, onde X pessoas trabalham semanalmente utilizando um sistema web para gerenciar as demandas na área de tecnologia da informação.

Após o uso da nova tecnologia pelos integrantes do setor durante 1 dia, será fornecido um formulário com questões para a avaliação do dispositivo. A análise do formulário visa quantificar a satisfação qualitativamente, e com isso avaliar a adequação para o caso específico.

A discussão dos dados obtidos, bem como das impressões capturadas durante o trabalho são então discutidas no tópico resultados.