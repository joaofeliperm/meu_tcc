\chapter{CONCLUSÃO}

\section{Objetivos alcançados}

Após todo o procedimento de testes de implantação e uso do hardware Raspberry Pi no ambiente de produção do estudo de caso, e posterior análise dos questionários preenchidos pelos usuários convidados para realizar o uso do equipamento, conclui-se que ele é uma boa alternativa em pequenas empresas.
Os custos discutidos deixam claro que o equipamento é uma oportunidade de diminuir o gasto para se montar sistemas computacionais, pois, como ficou evidenciado neste trabalho, o preço de um Raspberry Pi é, aproximadamente, 39\% menor que um computador desktop.

As configurações apresentadas demonstram que o equipamento é suficientemente flexível para atender a diversas necessidades comuns, bem como inovadoras.
A pergunta 9, considerada a mais importante do estudo de caso obteve um percentual de aprovação de 92\% dos respondentes, corroborando com as hipóteses de vantagem no uso do equipamento. Sendo assim, nota-se que a totalidade dos usuários aprovou o uso do Raspberry Pi no seu ambiente de produção.

Conclui-se, portanto, que a tecnologia apresentada neste trabalho foi considerada perfeitamente capaz de ser utilizada como meio de manipulação do sistema informatizado do setor, permitindo formar expectativas em relação à possibilidade de que o equipamento apresentado possa atingir uma visibilidade e usabilidade maiores em pequenas empresas, visto que muitas dessas organizações deixam de informatizar o seu negócio por causa dos elevados custos de compra e manutenção de computadores.

\section{Trabalhos Futuros}

Este trabalho mostrou que o Raspberry Pi obteve um bom desempenho quando utilizado para acessar e manipular sistemas desenvolvidos para a web, entretando sabe-se que existem sistemas desktop e que eles utilizam os recursos do equipamento onde fica hospedado. Sendo assim, seria interessante conhecer os limites impostos pelo Raspberry Pi quanto à instalação e utilização de sistemas com tais características.