\chapter{CONFIGURAÇÕES POSSÍVEIS}

Neste capítulo serão apresentados os periféricos utilizados juntamente com o Raspberry Pi no momento da realização do estudo de caso, outros acessórios que podem trabalhar com o equipamento e um comparativo de preços entre sistemas computacionais montados com o Pi e com um computador desktop.

\section{Periféricos utilizados no momento do estudo de caso}

O monitor utilizado foi o modelo FLATRON W2353V da marca LG. Para fazer a comunicação entre o monitor e a placa, foi empregado um cabo HDMI normal, sem qualquer adaptação.

Tanto o teclado como o mouse usados eram USB. O teclado era com fio e o mouse era sem.

A conexão com a internet foi estabelecida via cabo, que foi plugado à placa através do conector RJ45.

Para receber a instalação do sistema operacional, bem como armazenar dados, ou seja, para exercer a função de um HD, foi utilizado um cartão de memória MicroSD Classe 4 de 8GB da marca SanDisk. Para conectá-lo à placa foi empregado um adaptador SD para cartões MicroSD.

A fonte de alimentação empregada foi um carregador de celular que fornecia uma saída de 5V de tensão e uma corrente de 0,7A. Quando utilizada uma fonte de alimentação com o fornecimento de corrente menor que 0,7A, o Raspberry Pi até ligou, porém o teclado apresentou falhas no funcionamento, ora trabalhando normalmente, ora não.

\section{Outros acessórios}

\textbf{Dissipador de calor} Um dissipador de calor é um pequeno objeto de metal, normalmente com aletas, para criar bastante área de superfície para dissipar o calor de forma eficiente. Dissipadores de calor podem ser anexados aos chips que possam ficar quentes. O chipset do Raspberry Pi foi projetado para aplicações móveis, de modo que um dissipador de calor não é necessário na maioria das vezes. No entanto, como veremos mais tarde, existem casos em que você pode querer executar o Pi em altas velocidades ou processar números por um longo período, e o chip poderá então aquecer um pouco. Algumas pessoas relataram que o chip de rede pode ficar quente também.

\textbf{Relógio em tempo real} Você pode querer adicionar um chip de relógio em tempo real (como o DS1307) para logging ou marcação de hora enquanto estiver offline (desconectado).

\textbf{Módulo de câmera} Um módulo oficial de câmera Raspberry Pi de 5 megapixels já está disponível. É possível utilizar uma webcam USB.

\textbf{Display LCD} Muitos LCDs podem ser utilizados por meio de algumas conexões nos pinos GPIO.

\textbf{USB Wi-Fi} Muitos adaptadores externos USB Wi-Fi funcionam com o Pi; procure um que não consuma muita energia. 

No site http://elinux.org/RPi\_VerifiedPeripherals existe uma lista bem mais completa que a apresentada acima e que mostra diversos outros equipamentos compatíveis com o Raspberry Pi.

\newpage

\section{Comparativo de preços}

A tabela abaixo apresenta o comparativo de preços entres sitemas computacionais montados com Raspberry Pi e um computador desktop. As pesquisas de preços foram feitas nos sites farnellnewark.com.br, lojasamericanas.com, pontofrio.com.br e mercadolivre.com.br às 14:28h do dia 28 de setembro de 2013.

\begin{table}[!htpb]
 \centering
    \begin{tabular}{|p{4cm}|p{2cm}|p{4cm}|p{4cm}|} 
    \hline
        \textbf{Periféricos} & \textbf{Preço} & \textbf{Sistema Computacional com Raspberry Pi} & \textbf{Sistema Computacional com desktop} \\
    \hline
        Monitor c/ entrada HDMI (1) & R\$ 404,10 & X & X \\
    \hline
        Cabo HDMI (1) & R\$ 14,90 & X & X \\
    \hline
        Mouse USB (1) & R\$ 9,99 & X & \\
    \hline
        Teclado USB (2) & R\$ 22,40 & X & \\
    \hline
        Cartão MicroSD 8GB com adaptador (2) & R\$ 32,52 & X & \\
    \hline
        Carregador Micro USB (fonte de alimentação) (4) & R\$ 39,90 & X & \\
    \hline
        Cabo de força (fonte de alimentação) (4) & R\$ 5,30 & & X \\
    \hline
        Placa Raspberry Pi (3) & R\$ 176,02 & X & \\
    \hline
        Desktop (mouse e teclado inclusos) (1) & R\$ 719,10 & & X \\
    \hline
        \textbf{TOTAL} & & \textbf{R\$ 699,83} & \textbf{R\$ 1.143,40} \\
    \hline
    \end{tabular}
    \caption{Comparativo entre sistemas computacionais}
    \label{t_fixa}
\end{table}


Legenda:

1 - lojasamericanas.com 

2 - pontofrio.com.br 

3 - farnellnewark.com.br

4 - mercadolivre.com.br